% Options for packages loaded elsewhere
\PassOptionsToPackage{unicode}{hyperref}
\PassOptionsToPackage{hyphens}{url}
%
\documentclass[
]{article}
\usepackage{amsmath,amssymb}
\usepackage{lmodern}
\usepackage{iftex}
\ifPDFTeX
  \usepackage[T1]{fontenc}
  \usepackage[utf8]{inputenc}
  \usepackage{textcomp} % provide euro and other symbols
\else % if luatex or xetex
  \usepackage{unicode-math}
  \defaultfontfeatures{Scale=MatchLowercase}
  \defaultfontfeatures[\rmfamily]{Ligatures=TeX,Scale=1}
\fi
% Use upquote if available, for straight quotes in verbatim environments
\IfFileExists{upquote.sty}{\usepackage{upquote}}{}
\IfFileExists{microtype.sty}{% use microtype if available
  \usepackage[]{microtype}
  \UseMicrotypeSet[protrusion]{basicmath} % disable protrusion for tt fonts
}{}
\makeatletter
\@ifundefined{KOMAClassName}{% if non-KOMA class
  \IfFileExists{parskip.sty}{%
    \usepackage{parskip}
  }{% else
    \setlength{\parindent}{0pt}
    \setlength{\parskip}{6pt plus 2pt minus 1pt}}
}{% if KOMA class
  \KOMAoptions{parskip=half}}
\makeatother
\usepackage{xcolor}
\usepackage[margin=1in]{geometry}
\usepackage{color}
\usepackage{fancyvrb}
\newcommand{\VerbBar}{|}
\newcommand{\VERB}{\Verb[commandchars=\\\{\}]}
\DefineVerbatimEnvironment{Highlighting}{Verbatim}{commandchars=\\\{\}}
% Add ',fontsize=\small' for more characters per line
\usepackage{framed}
\definecolor{shadecolor}{RGB}{248,248,248}
\newenvironment{Shaded}{\begin{snugshade}}{\end{snugshade}}
\newcommand{\AlertTok}[1]{\textcolor[rgb]{0.94,0.16,0.16}{#1}}
\newcommand{\AnnotationTok}[1]{\textcolor[rgb]{0.56,0.35,0.01}{\textbf{\textit{#1}}}}
\newcommand{\AttributeTok}[1]{\textcolor[rgb]{0.77,0.63,0.00}{#1}}
\newcommand{\BaseNTok}[1]{\textcolor[rgb]{0.00,0.00,0.81}{#1}}
\newcommand{\BuiltInTok}[1]{#1}
\newcommand{\CharTok}[1]{\textcolor[rgb]{0.31,0.60,0.02}{#1}}
\newcommand{\CommentTok}[1]{\textcolor[rgb]{0.56,0.35,0.01}{\textit{#1}}}
\newcommand{\CommentVarTok}[1]{\textcolor[rgb]{0.56,0.35,0.01}{\textbf{\textit{#1}}}}
\newcommand{\ConstantTok}[1]{\textcolor[rgb]{0.00,0.00,0.00}{#1}}
\newcommand{\ControlFlowTok}[1]{\textcolor[rgb]{0.13,0.29,0.53}{\textbf{#1}}}
\newcommand{\DataTypeTok}[1]{\textcolor[rgb]{0.13,0.29,0.53}{#1}}
\newcommand{\DecValTok}[1]{\textcolor[rgb]{0.00,0.00,0.81}{#1}}
\newcommand{\DocumentationTok}[1]{\textcolor[rgb]{0.56,0.35,0.01}{\textbf{\textit{#1}}}}
\newcommand{\ErrorTok}[1]{\textcolor[rgb]{0.64,0.00,0.00}{\textbf{#1}}}
\newcommand{\ExtensionTok}[1]{#1}
\newcommand{\FloatTok}[1]{\textcolor[rgb]{0.00,0.00,0.81}{#1}}
\newcommand{\FunctionTok}[1]{\textcolor[rgb]{0.00,0.00,0.00}{#1}}
\newcommand{\ImportTok}[1]{#1}
\newcommand{\InformationTok}[1]{\textcolor[rgb]{0.56,0.35,0.01}{\textbf{\textit{#1}}}}
\newcommand{\KeywordTok}[1]{\textcolor[rgb]{0.13,0.29,0.53}{\textbf{#1}}}
\newcommand{\NormalTok}[1]{#1}
\newcommand{\OperatorTok}[1]{\textcolor[rgb]{0.81,0.36,0.00}{\textbf{#1}}}
\newcommand{\OtherTok}[1]{\textcolor[rgb]{0.56,0.35,0.01}{#1}}
\newcommand{\PreprocessorTok}[1]{\textcolor[rgb]{0.56,0.35,0.01}{\textit{#1}}}
\newcommand{\RegionMarkerTok}[1]{#1}
\newcommand{\SpecialCharTok}[1]{\textcolor[rgb]{0.00,0.00,0.00}{#1}}
\newcommand{\SpecialStringTok}[1]{\textcolor[rgb]{0.31,0.60,0.02}{#1}}
\newcommand{\StringTok}[1]{\textcolor[rgb]{0.31,0.60,0.02}{#1}}
\newcommand{\VariableTok}[1]{\textcolor[rgb]{0.00,0.00,0.00}{#1}}
\newcommand{\VerbatimStringTok}[1]{\textcolor[rgb]{0.31,0.60,0.02}{#1}}
\newcommand{\WarningTok}[1]{\textcolor[rgb]{0.56,0.35,0.01}{\textbf{\textit{#1}}}}
\usepackage{longtable,booktabs,array}
\usepackage{calc} % for calculating minipage widths
% Correct order of tables after \paragraph or \subparagraph
\usepackage{etoolbox}
\makeatletter
\patchcmd\longtable{\par}{\if@noskipsec\mbox{}\fi\par}{}{}
\makeatother
% Allow footnotes in longtable head/foot
\IfFileExists{footnotehyper.sty}{\usepackage{footnotehyper}}{\usepackage{footnote}}
\makesavenoteenv{longtable}
\usepackage{graphicx}
\makeatletter
\def\maxwidth{\ifdim\Gin@nat@width>\linewidth\linewidth\else\Gin@nat@width\fi}
\def\maxheight{\ifdim\Gin@nat@height>\textheight\textheight\else\Gin@nat@height\fi}
\makeatother
% Scale images if necessary, so that they will not overflow the page
% margins by default, and it is still possible to overwrite the defaults
% using explicit options in \includegraphics[width, height, ...]{}
\setkeys{Gin}{width=\maxwidth,height=\maxheight,keepaspectratio}
% Set default figure placement to htbp
\makeatletter
\def\fps@figure{htbp}
\makeatother
\setlength{\emergencystretch}{3em} % prevent overfull lines
\providecommand{\tightlist}{%
  \setlength{\itemsep}{0pt}\setlength{\parskip}{0pt}}
\setcounter{secnumdepth}{-\maxdimen} % remove section numbering
\ifLuaTeX
  \usepackage{selnolig}  % disable illegal ligatures
\fi
\IfFileExists{bookmark.sty}{\usepackage{bookmark}}{\usepackage{hyperref}}
\IfFileExists{xurl.sty}{\usepackage{xurl}}{} % add URL line breaks if available
\urlstyle{same} % disable monospaced font for URLs
\hypersetup{
  pdftitle={Biostat 200C Final},
  pdfauthor={FirstName LastName (UID XXX-XXX-XXX)},
  hidelinks,
  pdfcreator={LaTeX via pandoc}}

\title{Biostat 200C Final}
\usepackage{etoolbox}
\makeatletter
\providecommand{\subtitle}[1]{% add subtitle to \maketitle
  \apptocmd{\@title}{\par {\large #1 \par}}{}{}
}
\makeatother
\subtitle{Due June 16, 2023 @ 11:59PM}
\author{FirstName LastName (UID XXX-XXX-XXX)}
\date{}

\begin{document}
\maketitle

{
\setcounter{tocdepth}{4}
\tableofcontents
}
This is an open book test. Helping or asking help from others is
considered plagiarism.

\hypertarget{q1.-25-pts-survival-data-analysis}{%
\subsection{Q1. (25 pts) Survival data
analysis}\label{q1.-25-pts-survival-data-analysis}}

Consider following survival times of 25 patients with no history of
chronic diesease (\texttt{chr\ =\ 0}) and 25 patients with history of
chronic disease (\texttt{chr\ =\ 1}).

\begin{enumerate}
\def\labelenumi{\arabic{enumi}.}
\tightlist
\item
  Manually fill in the missing information in the following tables of
  ordered failure times for groups 1 (\texttt{chr\ =\ 0}) and 2
  (\texttt{chr\ =\ 1}). Explain how survival probabilities (last column)
  are calculated.
\end{enumerate}

Group 1 (\texttt{chr\ =\ 0}):

\begin{longtable}[]{@{}llll@{}}
\toprule()
time & n.risk & n.event & survival \\
\midrule()
\endhead
1.8 & 25 & 1 & 0.96 \\
2.2 & 24 & 1 & 0.92 \\
2.5 & 23 & 1 & 0.88 \\
2.6 & 22 & 1 & 0.84 \\
3.0 & 21 & 1 & 0.80 \\
3.5 & 20 & \textbf{???} & \textbf{???} \\
3.8 & 19 & 1 & 0.72 \\
5.3 & 18 & 1 & 0.68 \\
5.4 & 17 & 1 & 0.64 \\
5.7 & 16 & 1 & 0.60 \\
6.6 & 15 & 1 & 0.56 \\
8.2 & 14 & 1 & 0.52 \\
8.7 & 13 & 1 & 0.48 \\
9.2 & \textbf{???} & \textbf{???} & \textbf{???} \\
9.8 & 10 & 1 & 0.36 \\
10.0 & 9 & 1 & 0.32 \\
10.2 & 8 & 1 & 0.28 \\
10.7 & 7 & 1 & 0.24 \\
11.0 & 6 & 1 & 0.20 \\
11.1 & 5 & 1 & 0.16 \\
11.7 & 4 & \textbf{???} & \textbf{???} \\
\bottomrule()
\end{longtable}

Group 2 (\texttt{chr\ =\ 1}):

\begin{longtable}[]{@{}llll@{}}
\toprule()
time & n.risk & n.event & survival \\
\midrule()
\endhead
1.4 & 25 & 1 & 0.96 \\
1.6 & 24 & 1 & 0.92 \\
1.8 & 23 & 1 & 0.88 \\
2.4 & 22 & 1 & 0.84 \\
2.8 & 21 & 1 & 0.80 \\
2.9 & 20 & 1 & 0.76 \\
3.1 & 19 & 1 & 0.72 \\
3.5 & 18 & 1 & 0.68 \\
3.6 & 17 & 1 & 0.64 \\
3.9 & \textbf{???} & \textbf{???} & \textbf{???} \\
4.1 & \textbf{???} & \textbf{???} & \textbf{???} \\
4.2 & \textbf{???} & \textbf{???} & \textbf{???} \\
4.7 & 13 & 1 & 0.48 \\
4.9 & 12 & 1 & 0.44 \\
5.2 & 11 & 1 & 0.40 \\
5.8 & 10 & 1 & 0.36 \\
5.9 & 9 & 1 & 0.32 \\
6.5 & 8 & 1 & 0.28 \\
7.8 & 7 & 1 & 0.24 \\
8.3 & 6 & 1 & 0.20 \\
8.4 & 5 & 1 & 0.16 \\
8.8 & 4 & 1 & 0.12 \\
9.1 & \textbf{???} & \textbf{???} & 0.08 \\
9.9 & \textbf{???} & \textbf{???} & 0.04 \\
11.4 & 1 & 1 & 0.00 \\
\bottomrule()
\end{longtable}

\begin{enumerate}
\def\labelenumi{\arabic{enumi}.}
\setcounter{enumi}{1}
\item
  Use R to display the Kaplan-Meier survival curves for groups 1
  (\texttt{chr\ =\ 0}) and 2 (\texttt{chr\ =\ 1}).
\item
  Write down the log-likelihood of the parametric exponential
  (proportional hazard) model for survival times. Explain why this model
  can be fit as a generalized linear model with offset.
\item
  Fit the exponential (proportional hazard) model on the \texttt{chr}
  data using R. Interpret the coefficients.
\item
  Comment on the limitation of exponential model compared to other more
  flexible models such as Weibull.
\end{enumerate}

\hypertarget{q2.-35-pts-longitudinal-data-analysis}{%
\subsection{Q2. (35 pts) Longitudinal data
analysis}\label{q2.-35-pts-longitudinal-data-analysis}}

This question is adapted from Exercise 13.1 of ELMR (p294).

The \texttt{ohio} data concerns 536 children from Steubenville, Ohio and
were taken as part of a study on the effects of air pollution. Children
were in the study for 4 years from ages 7 to 10. The response was
whether they wheezed (difficulty with breathing) or not. The variables
are

\begin{itemize}
\item
  \texttt{resp}: an indicator of wheeze status (1 = yes, 0 = no)
\item
  \texttt{id}: an identifier for the child
\item
  \texttt{age}: 7 yrs = -2, 8 yrs = -1, 9 yrs = 0, 10 yrs = 1
\item
  \texttt{smoke}: an indicator of maternal smoking status at the first
  year of the study (1 = smoker, 0 = non-smoker).
\end{itemize}

\begin{enumerate}
\def\labelenumi{\arabic{enumi}.}
\item
  Construct a table that shows proportion of children who wheeze for 0,
  1, 2, 3 or 4 years broken down by maternal smoking status.
\item
  Make a plot which shows how the proportion of children wheezing
  changes by age with a separate line for smoking and nonsmoking
  mothers.
\item
  Group the data by child to count the total (out of four) years of
  wheezing. Fit a binomial GLM to this response to check for a maternal
  smoking effect. Does this prove there is a smoking effect or could
  there be another plausible explanation? Discuss the potential issue of
  using GLM here.
\item
  Fit a model for each individual response using a GLMM fit using
  penalized quasi-likelihood. Interpret the coefficients of age and
  maternal smoking. How do the odds of wheezing change numerically over
  time? Explain what you observe in the output AIC, BIC, and logLik.
\item
  Now fit the same model but using adaptive Gaussian-Hermit quadrature
  (with 25 knots). Compare to the previous model fit. Interpret the
  fixed effect coefficients. Interpret the variance component estimates.
  What is the odds ratio for wheezing comparing a smoking mother
  (\texttt{smoke\ =\ 1}) vs a non-smoking mother (\texttt{smoke\ =\ 0})
  and its 95\% confidence interval.
\item
  Fit the model using GEE. Use an autoregressive rather than
  exchangeable error structure. Explain the underlying assumptions of
  the autoregressive correlation structure. Does the GLMM approach in 5
  assumes the same correlation structure within each cluster/individual?
  Compare the results to the previous model fits. In your model, what
  indicates that a child who already wheezes is likely to continue to
  wheeze? What happens if we misspecified the correlation structure?
\item
  What is your overall conclusion regarding the effect of age and
  maternal smoking? Can we trust the GLM result or are the GLMM models
  preferable?
\end{enumerate}

\hypertarget{q3.-40-pts-lmm-and-gamm}{%
\subsection{Q3. (40 pts) LMM and GAMM}\label{q3.-40-pts-lmm-and-gamm}}

This question is adapted from Exercise 11.2 of ELMR (p251). Read the
documentation of the dataset \texttt{hprice} in Faraway package before
working on this problem.

\begin{enumerate}
\def\labelenumi{\arabic{enumi}.}
\item
  Make a plot of the data on a single panel to show how housing prices
  increase by year. Describe what can be seen in the plot.
\item
  Fit a linear model with the (log) house price as the response and all
  other variables (except msa) as fixed effect predictors. Which terms
  are statistically significant? Discuss the coefficient for time.
\item
  Make a plot that shows how per-capita income changes over time. What
  is the nature of the increase? Make a similar plot to show how income
  growth changes over time. Comment on the plot.
\item
  Create a new variable that is the per-capita income for the first time
  period for each MSA. Refit the same linear model but now using the
  initial income and not the income as it changes over time. Compare the
  two models.
\item
  Fit a mixed effects model that has a random intercept for each MSA.
  Why might this be reasonable? The rest of the model should have the
  same structure as in the previous question. Make a numerical
  interpretation of the coefficient of time in your model. Explain the
  difference between REML and MLE methods.
\item
  Fit a model that omits the adjacent to water and rent control
  predictors. Test whether this reduction in the model can be supported.
\item
  It is possible that the increase in prices may not be linear in year.
  Fit an additive mixed model where smooth is added to year. Make a plot
  to show how prices have increased over time.
\item
  Interpret the coefficients in the previous model for the initial
  annual income, growth and regulation predictors.
\end{enumerate}

\hypertarget{optional-extra-credit-problem}{%
\subsection{Optional Extra Credit
Problem*}\label{optional-extra-credit-problem}}

\begin{quote}
\begin{quote}
This problem is meant to offer another chance to demonstrate
understanding of some of the material on the mid-term. If you choose to
do this problem and your score is higher than your mid-term grade, then
your mid-term grade will be reweighted to be
\texttt{New\ Midterm\ Grade\ =\ .8*Old\ Midterm\ Grade\ +\ .2*Extra\ Credit\ Problem}
\end{quote}
\end{quote}

The following table shows numbers of beetles dead after five hours
exposure to gaseous carbon disulphide at various concentrations.

\begin{Shaded}
\begin{Highlighting}[]
\NormalTok{(beetle }\OtherTok{\textless{}{-}} \FunctionTok{tibble}\NormalTok{(}\AttributeTok{dose =} \FunctionTok{c}\NormalTok{(}\FloatTok{1.6907}\NormalTok{, }\FloatTok{1.7242}\NormalTok{, }\FloatTok{1.7552}\NormalTok{, }\FloatTok{1.7842}\NormalTok{, }\FloatTok{1.8113}\NormalTok{, }\FloatTok{1.8369}\NormalTok{, }\FloatTok{1.8610}\NormalTok{, }\FloatTok{1.8839}\NormalTok{),}
                 \AttributeTok{beetles =} \FunctionTok{c}\NormalTok{(}\DecValTok{59}\NormalTok{, }\DecValTok{60}\NormalTok{, }\DecValTok{62}\NormalTok{, }\DecValTok{56}\NormalTok{, }\DecValTok{63}\NormalTok{, }\DecValTok{59}\NormalTok{, }\DecValTok{62}\NormalTok{, }\DecValTok{60}\NormalTok{),}
                 \AttributeTok{killed =} \FunctionTok{c}\NormalTok{(}\DecValTok{6}\NormalTok{, }\DecValTok{13}\NormalTok{, }\DecValTok{18}\NormalTok{, }\DecValTok{28}\NormalTok{, }\DecValTok{52}\NormalTok{, }\DecValTok{53}\NormalTok{, }\DecValTok{61}\NormalTok{, }\DecValTok{60}\NormalTok{)))}
\end{Highlighting}
\end{Shaded}

\begin{verbatim}
## # A tibble: 8 x 3
##    dose beetles killed
##   <dbl>   <dbl>  <dbl>
## 1  1.69      59      6
## 2  1.72      60     13
## 3  1.76      62     18
## 4  1.78      56     28
## 5  1.81      63     52
## 6  1.84      59     53
## 7  1.86      62     61
## 8  1.88      60     60
\end{verbatim}

\begin{enumerate}
\def\labelenumi{\arabic{enumi}.}
\item
  Let \(x_i\) be \texttt{dose}, \(n_i\) be the number of beetles, and
  \(y_i\) be the number of killed. Plot the proportions
  \(p_i = y_i/n_i\) plotted against dose \(x_i\).
\item
  We fit a logistic model to understand the relationship between dose
  and the probably of being killed. Write out the logistic model and
  associated log-likelihood function.
\item
  Derive the scores, \(\mathbf{U}\), with respect to parameters in the
  above logistic model. (Hint there are two parameters)
\item
  Derive the information matrix, \(\mathcal{I}\) (Hint, a \(2\times 2\)
  matrix)
\item
  Maximum likelihood estimates are obtained by solving the iterative
  equation
\end{enumerate}

\[
\mathcal{I}^{(m-1)}\mathbf{b}^{(m)} = \mathcal{I}^{(m-1)}\mathbf{b}^{(m-1)}+ \mathbf{U}^{(m-1)}
\] where \(\mathbf{b}\) is the vector of estimates. Starting with
\(\mathbf{b}^{(0)} = 0\), implement this algorithm to show successive
iterations are

\begin{longtable}[]{@{}llll@{}}
\toprule()
Iterations & \(\beta_1\) & \(\beta_2\) & log-likelihood \\
\midrule()
\endhead
0 & 0 & 0 & -333.404 \\
1 & -37.856 & 21.337 & -200.010 \\
2 & -53.853 & 30.384 & -187.274 \\
3 & & & \\
4 & & & \\
5 & & & \\
6 & -60.717 & 34.270 & -186.235 \\
\bottomrule()
\end{longtable}

\begin{itemize}
\item
  If after 6 steps, the model converged. For this final model, calculate
  the deviance. What is the distribution the deviance has?
\item
  Does the model fit the data well? justify your answer.
\end{itemize}

\end{document}
