% Options for packages loaded elsewhere
\PassOptionsToPackage{unicode}{hyperref}
\PassOptionsToPackage{hyphens}{url}
%
\documentclass[
]{article}
\usepackage{amsmath,amssymb}
\usepackage{iftex}
\ifPDFTeX
  \usepackage[T1]{fontenc}
  \usepackage[utf8]{inputenc}
  \usepackage{textcomp} % provide euro and other symbols
\else % if luatex or xetex
  \usepackage{unicode-math} % this also loads fontspec
  \defaultfontfeatures{Scale=MatchLowercase}
  \defaultfontfeatures[\rmfamily]{Ligatures=TeX,Scale=1}
\fi
\usepackage{lmodern}
\ifPDFTeX\else
  % xetex/luatex font selection
\fi
% Use upquote if available, for straight quotes in verbatim environments
\IfFileExists{upquote.sty}{\usepackage{upquote}}{}
\IfFileExists{microtype.sty}{% use microtype if available
  \usepackage[]{microtype}
  \UseMicrotypeSet[protrusion]{basicmath} % disable protrusion for tt fonts
}{}
\makeatletter
\@ifundefined{KOMAClassName}{% if non-KOMA class
  \IfFileExists{parskip.sty}{%
    \usepackage{parskip}
  }{% else
    \setlength{\parindent}{0pt}
    \setlength{\parskip}{6pt plus 2pt minus 1pt}}
}{% if KOMA class
  \KOMAoptions{parskip=half}}
\makeatother
\usepackage{xcolor}
\usepackage[margin=1in]{geometry}
\usepackage{color}
\usepackage{fancyvrb}
\newcommand{\VerbBar}{|}
\newcommand{\VERB}{\Verb[commandchars=\\\{\}]}
\DefineVerbatimEnvironment{Highlighting}{Verbatim}{commandchars=\\\{\}}
% Add ',fontsize=\small' for more characters per line
\usepackage{framed}
\definecolor{shadecolor}{RGB}{248,248,248}
\newenvironment{Shaded}{\begin{snugshade}}{\end{snugshade}}
\newcommand{\AlertTok}[1]{\textcolor[rgb]{0.94,0.16,0.16}{#1}}
\newcommand{\AnnotationTok}[1]{\textcolor[rgb]{0.56,0.35,0.01}{\textbf{\textit{#1}}}}
\newcommand{\AttributeTok}[1]{\textcolor[rgb]{0.13,0.29,0.53}{#1}}
\newcommand{\BaseNTok}[1]{\textcolor[rgb]{0.00,0.00,0.81}{#1}}
\newcommand{\BuiltInTok}[1]{#1}
\newcommand{\CharTok}[1]{\textcolor[rgb]{0.31,0.60,0.02}{#1}}
\newcommand{\CommentTok}[1]{\textcolor[rgb]{0.56,0.35,0.01}{\textit{#1}}}
\newcommand{\CommentVarTok}[1]{\textcolor[rgb]{0.56,0.35,0.01}{\textbf{\textit{#1}}}}
\newcommand{\ConstantTok}[1]{\textcolor[rgb]{0.56,0.35,0.01}{#1}}
\newcommand{\ControlFlowTok}[1]{\textcolor[rgb]{0.13,0.29,0.53}{\textbf{#1}}}
\newcommand{\DataTypeTok}[1]{\textcolor[rgb]{0.13,0.29,0.53}{#1}}
\newcommand{\DecValTok}[1]{\textcolor[rgb]{0.00,0.00,0.81}{#1}}
\newcommand{\DocumentationTok}[1]{\textcolor[rgb]{0.56,0.35,0.01}{\textbf{\textit{#1}}}}
\newcommand{\ErrorTok}[1]{\textcolor[rgb]{0.64,0.00,0.00}{\textbf{#1}}}
\newcommand{\ExtensionTok}[1]{#1}
\newcommand{\FloatTok}[1]{\textcolor[rgb]{0.00,0.00,0.81}{#1}}
\newcommand{\FunctionTok}[1]{\textcolor[rgb]{0.13,0.29,0.53}{\textbf{#1}}}
\newcommand{\ImportTok}[1]{#1}
\newcommand{\InformationTok}[1]{\textcolor[rgb]{0.56,0.35,0.01}{\textbf{\textit{#1}}}}
\newcommand{\KeywordTok}[1]{\textcolor[rgb]{0.13,0.29,0.53}{\textbf{#1}}}
\newcommand{\NormalTok}[1]{#1}
\newcommand{\OperatorTok}[1]{\textcolor[rgb]{0.81,0.36,0.00}{\textbf{#1}}}
\newcommand{\OtherTok}[1]{\textcolor[rgb]{0.56,0.35,0.01}{#1}}
\newcommand{\PreprocessorTok}[1]{\textcolor[rgb]{0.56,0.35,0.01}{\textit{#1}}}
\newcommand{\RegionMarkerTok}[1]{#1}
\newcommand{\SpecialCharTok}[1]{\textcolor[rgb]{0.81,0.36,0.00}{\textbf{#1}}}
\newcommand{\SpecialStringTok}[1]{\textcolor[rgb]{0.31,0.60,0.02}{#1}}
\newcommand{\StringTok}[1]{\textcolor[rgb]{0.31,0.60,0.02}{#1}}
\newcommand{\VariableTok}[1]{\textcolor[rgb]{0.00,0.00,0.00}{#1}}
\newcommand{\VerbatimStringTok}[1]{\textcolor[rgb]{0.31,0.60,0.02}{#1}}
\newcommand{\WarningTok}[1]{\textcolor[rgb]{0.56,0.35,0.01}{\textbf{\textit{#1}}}}
\usepackage{longtable,booktabs,array}
\usepackage{calc} % for calculating minipage widths
% Correct order of tables after \paragraph or \subparagraph
\usepackage{etoolbox}
\makeatletter
\patchcmd\longtable{\par}{\if@noskipsec\mbox{}\fi\par}{}{}
\makeatother
% Allow footnotes in longtable head/foot
\IfFileExists{footnotehyper.sty}{\usepackage{footnotehyper}}{\usepackage{footnote}}
\makesavenoteenv{longtable}
\usepackage{graphicx}
\makeatletter
\def\maxwidth{\ifdim\Gin@nat@width>\linewidth\linewidth\else\Gin@nat@width\fi}
\def\maxheight{\ifdim\Gin@nat@height>\textheight\textheight\else\Gin@nat@height\fi}
\makeatother
% Scale images if necessary, so that they will not overflow the page
% margins by default, and it is still possible to overwrite the defaults
% using explicit options in \includegraphics[width, height, ...]{}
\setkeys{Gin}{width=\maxwidth,height=\maxheight,keepaspectratio}
% Set default figure placement to htbp
\makeatletter
\def\fps@figure{htbp}
\makeatother
\setlength{\emergencystretch}{3em} % prevent overfull lines
\providecommand{\tightlist}{%
  \setlength{\itemsep}{0pt}\setlength{\parskip}{0pt}}
\setcounter{secnumdepth}{-\maxdimen} % remove section numbering
\ifLuaTeX
  \usepackage{selnolig}  % disable illegal ligatures
\fi
\usepackage{bookmark}
\IfFileExists{xurl.sty}{\usepackage{xurl}}{} % add URL line breaks if available
\urlstyle{same}
\hypersetup{
  pdfauthor={Name \& UID: \_\_\_\_\_\_\_\_\_\_\_\_\_\_\_\_\_\_\_\_\_\_\_\_\_\_\_\_\_\_\_\_\_\_\_\_\_\_\_\_},
  hidelinks,
  pdfcreator={LaTeX via pandoc}}

\title{Biostat 200C Midterm\\
Total points: 100 pts}
\usepackage{etoolbox}
\makeatletter
\providecommand{\subtitle}[1]{% add subtitle to \maketitle
  \apptocmd{\@title}{\par {\large #1 \par}}{}{}
}
\makeatother
\subtitle{May 16, 2024 @ 10am-12pm}
\author{Name \& UID:
\_\_\_\_\_\_\_\_\_\_\_\_\_\_\_\_\_\_\_\_\_\_\_\_\_\_\_\_\_\_\_\_\_\_\_\_\_\_\_\_}
\date{}

\begin{document}
\maketitle

{
\setcounter{tocdepth}{4}
\tableofcontents
}
\section{Directions}\label{directions}

The exam is allotted for 2 hours from 10 AM to 12 PM. This is a
closed-book exam, and the use of computers, cell-phones, and written
materials are prohibited. You are allowed to use a calculator as you
need to answer problems.

There are 10 total questions on the exam with multiple parts. For true
and false questions, clearly circle your selected answer for each
problem. For written short answers, write your answers starting below
the question prompt, or on the next page if there is not enough space.
Clearly label each part of the problem. Additional scratch paper will be
provided should you need any.

\textbf{BE SURE TO INCLUDE YOUR NAME ON ANY LOOSE PAPERS YOU INTEND TO
SUBMIT.}

Good luck on the exam!

\newpage

\subsection{Q1. (8 pts) True or false}\label{q1.-8-pts-true-or-false}

\begin{longtable}[]{@{}
  >{\raggedright\arraybackslash}p{(\columnwidth - 6\tabcolsep) * \real{0.0645}}
  >{\raggedright\arraybackslash}p{(\columnwidth - 6\tabcolsep) * \real{0.0645}}
  >{\raggedright\arraybackslash}p{(\columnwidth - 6\tabcolsep) * \real{0.0645}}
  >{\raggedright\arraybackslash}p{(\columnwidth - 6\tabcolsep) * \real{0.8065}}@{}}
\toprule\noalign{}
\begin{minipage}[b]{\linewidth}\raggedright
No.
\end{minipage} & \begin{minipage}[b]{\linewidth}\raggedright
\end{minipage} & \begin{minipage}[b]{\linewidth}\raggedright
\end{minipage} & \begin{minipage}[b]{\linewidth}\raggedright
Statement
\end{minipage} \\
\midrule\noalign{}
\endhead
\bottomrule\noalign{}
\endlastfoot
1. & T & F & Suppose the outcome in a study is systolic blood pressure,
treated continuously. Then a logistic model should be used for
analysis. \\
2. & T & F & The range of the logistic function lies between 0 and 1. \\
3. & T & F & The shape of the logistic function is linear. \\
4. & T & F & The logistic model can be used to describes the probability
of disease development, e.g., risk for the disease, for a given set of
independent variables. \\
5. & T & F & The logistic model can only be applied to analyze a
prospective (or follow-up) study. \\
6. & T & F & Given a fitted logistic model from retrospective
(case-control) study, we can estimate the disease risk for a specific
individual. \\
7. & T & F & Given a fitted logistic model from a prospective
(follow-up) study, it is not possible to estimate individual risk as the
constant term (intercept) cannot be estimated. \\
8. & T & F & Given a fitted logistic model from a retrospective
(case-control) study, an odds ratio can be estimated. \\
9. & T & F & The coefficient \(\beta_j\) in the logistic model can be
interpreted as the change in log odds corresponding to a one unit change
in the variable \(x_j\) that ignores the contribution of other
variables. \\
10. & T & F & When estimating the parameters of the logistic model,
least squares estimation is the preferred method of estimation. \\
11. & T & F & The likelihood function \(L(\boldsymbol{\theta})\)
represents the joint probability of observing the data that has been
collected for analysis. \\
12. & T & F & The maximum likelihood estimate (MLE) maximizes the
function \(L(\boldsymbol{\theta})\). \\
13. & T & F & The analysis of deviance (or likelihood ratio test) is the
preferred method for testing hypotheses about parameters in the logistic
model (assuming asymptotic assumptions are reasonably met). \\
14. & T & F & If a likelihood function for a logistic model contains
five parameters, then the ML solution optimizes a function with five
unknowns using an iterative procedure. \\
15. & T & F & The analysis of of deviance (or likelihood ratio test)
uses the maximized likelihood value in its computation. \\
16. & T & F & The Wald test (or z test) and the likelihood ratio test of
the same hypothesis give approximately the same results in large
samples. \\
\end{longtable}

\newpage

\subsection{Q2. (20 pts) Binomial
model}\label{q2.-20-pts-binomial-model}

Consider a data set where the outcome is count of coronary heart disease
(\texttt{chd}) cases. \texttt{total} is the total number of study
individuals in each predictor profile. The predictors of interest are

\begin{longtable}[]{@{}ll@{}}
\toprule\noalign{}
Variable & Type \\
\midrule\noalign{}
\endhead
\bottomrule\noalign{}
\endlastfoot
\texttt{sex} & binary, 1 = male, 0 = female \\
\texttt{age} & binary, 1 = age \textgreater{} 55, 0 = age \(\le 55\) \\
\texttt{ecg} & binary, 1 = abnormal, 0 = normal \\
\end{longtable}

\begin{verbatim}
##   cases total sex age ecg
## 1    17   274   0   0   0
## 2    15   122   0   1   0
## 3     7    59   0   0   1
## 4     5    32   0   1   1
## 5     1     8   1   0   0
## 6     9    39   1   1   0
## 7     3    17   1   0   1
## 8    14    58   1   1   1
\end{verbatim}

Researchers are interested in whether these predictors are associated
with the chance of having \texttt{chd}. Based on the R output, answer
following questions.

\begin{Shaded}
\begin{Highlighting}[]
\NormalTok{lmod }\OtherTok{\textless{}{-}} \FunctionTok{glm}\NormalTok{(}\FunctionTok{cbind}\NormalTok{(cases, total }\SpecialCharTok{{-}}\NormalTok{ cases) }\SpecialCharTok{\textasciitilde{}}\NormalTok{ ., }\AttributeTok{family =}\NormalTok{ binomial, }\AttributeTok{data =}\NormalTok{ chd)}
\FunctionTok{summary}\NormalTok{(lmod)}
\end{Highlighting}
\end{Shaded}

\begin{verbatim}
## 
## Call:
## glm(formula = cbind(cases, total - cases) ~ ., family = binomial, 
##     data = chd)
## 
## Coefficients:
##             Estimate Std. Error z value Pr(>|z|)    
## (Intercept)  -2.6163     0.2123 -12.322   <2e-16 ***
## sex1          0.6223     0.3194   1.948   0.0514 .  
## age1          0.6157     0.2838   2.169   0.0301 *  
## ecg1          0.3620     0.2904   1.247   0.2126    
## ---
## Signif. codes:  0 '***' 0.001 '**' 0.01 '*' 0.05 '.' 0.1 ' ' 1
## 
## (Dispersion parameter for binomial family taken to be 1)
## 
##     Null deviance: 21.33202  on 7  degrees of freedom
## Residual deviance:  0.95443  on 4  degrees of freedom
## AIC: 37.666
## 
## Number of Fisher Scoring iterations: 4
\end{verbatim}

\newpage

1 (3 pts). Using the R output above, what are the deviance and log
likelihood of the fitted model?

3 (2 pts). Does this model with three predictors provide a better fit
than the intercept only model? State the null hypothesis, the test
statistic and its degrees of freedom, and the conclusion of the test.

4 (3 pts). Calculate the odds (chd vs.~no chd) for a 60-year-old male
with abnormal ecg.

5 (3 pts). Suppose we fit a logistic model with intercept, predictors
\texttt{sex}, \texttt{age} and \texttt{ecg}, their pairwise 2-way
interactions, and the 3-way interaction \texttt{sex\ x\ age\ x\ ecg}.
How many parameters are in this model? What will be the deviance of this
model? Explain.

6 (4 pts). What is the Hessian matrix of the log-likelihood function of
a logistic model? Explain why the log-likelihood function of a logistic
model is a concave function. Is it still concave if we used the probit
link function?

7 (2 pts). Suppose the data is recorded at individual level. Each row is
a person and the \texttt{case} variable indicates whether that person
has coronary heart disease (\texttt{yes}) or not (\texttt{no}). If we
fit a logistic regression using \texttt{chd} as the binary outcome, what
will be the fitted regression coefficients for \texttt{sex},
\texttt{age}, and \texttt{ecg}? Explain your answer.

\begin{verbatim}
## # A tibble: 609 x 4
##    sex   age   ecg   chd  
##    <fct> <fct> <fct> <chr>
##  1 0     0     0     yes  
##  2 0     0     0     yes  
##  3 0     0     0     yes  
##  4 0     0     0     yes  
##  5 0     0     0     yes  
##  6 0     0     0     yes  
##  7 0     0     0     yes  
##  8 0     0     0     yes  
##  9 0     0     0     yes  
## 10 0     0     0     yes  
## # i 599 more rows
\end{verbatim}

\newpage

\subsection{Q2: (Extra Space)}\label{q2-extra-space}

\newpage

\subsection{Q3. (10 pts) ROC, sensitivity,
specificity}\label{q3.-10-pts-roc-sensitivity-specificity}

Recall the heart disease data set \texttt{wcgs} in ELMR. We model the
dependence of disease status \texttt{chd} on \texttt{height} and number
of cigarettes per day \texttt{cigs}. Assume we use a cut-point of 0.15
in classification of disease status.

\begin{Shaded}
\begin{Highlighting}[]
\NormalTok{lmod }\OtherTok{\textless{}{-}} \FunctionTok{glm}\NormalTok{(chd }\SpecialCharTok{\textasciitilde{}}\NormalTok{ height }\SpecialCharTok{+}\NormalTok{ cigs, }\AttributeTok{family =}\NormalTok{ binomial, }\AttributeTok{data =}\NormalTok{ wcgs)}
\FunctionTok{summary}\NormalTok{(lmod)}
\end{Highlighting}
\end{Shaded}

\begin{verbatim}
## 
## Call:
## glm(formula = chd ~ height + cigs, family = binomial, data = wcgs)
## 
## Coefficients:
##             Estimate Std. Error z value Pr(>|z|)    
## (Intercept) -4.50161    1.84186  -2.444   0.0145 *  
## height       0.02521    0.02633   0.957   0.3383    
## cigs         0.02313    0.00404   5.724 1.04e-08 ***
## ---
## Signif. codes:  0 '***' 0.001 '**' 0.01 '*' 0.05 '.' 0.1 ' ' 1
## 
## (Dispersion parameter for binomial family taken to be 1)
## 
##     Null deviance: 1781.2  on 3153  degrees of freedom
## Residual deviance: 1749.0  on 3151  degrees of freedom
## AIC: 1755
## 
## Number of Fisher Scoring iterations: 5
\end{verbatim}

1 (2 pts). Given, the numbers below, find the number of true positive
(TP), false positive (FP), true negatives (TN), and false negatives
(FN).

\begin{Shaded}
\begin{Highlighting}[]
\NormalTok{wcgs }\SpecialCharTok{\%\textgreater{}\%}
  \FunctionTok{mutate}\NormalTok{(}\AttributeTok{predprob =} \FunctionTok{predict}\NormalTok{(lmod, }\AttributeTok{type =} \StringTok{"response"}\NormalTok{)) }\SpecialCharTok{\%\textgreater{}\%}
  \FunctionTok{mutate}\NormalTok{(}\AttributeTok{predout =} \FunctionTok{ifelse}\NormalTok{(predprob }\SpecialCharTok{\textgreater{}=} \FloatTok{0.15}\NormalTok{, }\StringTok{"yes"}\NormalTok{, }\StringTok{"no"}\NormalTok{)) }\SpecialCharTok{\%\textgreater{}\%}
  \FunctionTok{xtabs}\NormalTok{(}\SpecialCharTok{\textasciitilde{}}\NormalTok{ chd }\SpecialCharTok{+}\NormalTok{ predout, }\AttributeTok{data =}\NormalTok{ .)}
\end{Highlighting}
\end{Shaded}

\begin{verbatim}
##      predout
## chd     no  yes
##   no  2838   59
##   yes  245   12
\end{verbatim}

2 (2 pts). Compute the sensitivity, specificty, and 1 - specificity.

3 (3 pts). Suppose we used a cut-point of 0.25 instead of 0.15 when
classifying disease status. Explain what effect, if any, this would have
on the values computed in (2).

4 (3 pts). Explain what the area under an ROC curve is and why it is a
measure of the classification performance of a logistic model.

\newpage

\subsection{Q3: (Extra Space)}\label{q3-extra-space}

\newpage

\subsection{Q4. (9 pts) Matched case-control
study}\label{q4.-9-pts-matched-case-control-study}

Consider a matched pair study (from ELMR) of the relationship between
X-ray exposure and childhood acute myeloid leukemia. Matching was done
on age, race and county of residence.

\begin{Shaded}
\begin{Highlighting}[]
\NormalTok{ramlxray }\OtherTok{\textless{}{-}} \FunctionTok{as\_tibble}\NormalTok{(amlxray) }\SpecialCharTok{\%\textgreater{}\%}
  \FunctionTok{select}\NormalTok{(ID, disease, Sex, age, Fray, CnRay) }\SpecialCharTok{\%\textgreater{}\%}
  \FunctionTok{mutate}\NormalTok{(}\AttributeTok{CnRay =} \FunctionTok{unclass}\NormalTok{(CnRay)) }\SpecialCharTok{\%\textgreater{}\%}
  \CommentTok{\# take out Down symdrome cases}
  \FunctionTok{filter}\NormalTok{(}\SpecialCharTok{!}\NormalTok{(ID }\SpecialCharTok{\%in\%} \FunctionTok{c}\NormalTok{(}\DecValTok{7010}\NormalTok{, }\DecValTok{7018}\NormalTok{, }\DecValTok{7066}\NormalTok{, }\DecValTok{7077}\NormalTok{, }\DecValTok{7146}\NormalTok{, }\DecValTok{7176}\NormalTok{, }\DecValTok{7189}\NormalTok{))) }\SpecialCharTok{\%\textgreater{}\%}
  \FunctionTok{print}\NormalTok{(}\AttributeTok{width =} \ConstantTok{Inf}\NormalTok{)}
\end{Highlighting}
\end{Shaded}

\begin{verbatim}
## # A tibble: 223 x 6
##    ID    disease Sex     age Fray  CnRay
##    <fct>   <dbl> <fct> <int> <fct> <int>
##  1 7004        1 F         0 no        1
##  2 7004        0 F         0 no        1
##  3 7006        1 M         6 no        3
##  4 7006        0 M         6 no        2
##  5 7009        1 F         8 no        1
##  6 7009        0 F         8 no        1
##  7 7013        1 M         4 no        1
##  8 7013        0 M         4 no        2
##  9 7014        1 F         9 no        1
## 10 7014        0 M         9 no        1
## # i 213 more rows
\end{verbatim}

The fitted coefficients are

\begin{Shaded}
\begin{Highlighting}[]
\FunctionTok{clogit}\NormalTok{(disease }\SpecialCharTok{\textasciitilde{}}\NormalTok{ Sex }\SpecialCharTok{+}\NormalTok{ Fray }\SpecialCharTok{+} \FunctionTok{unclass}\NormalTok{(CnRay) }\SpecialCharTok{+} \FunctionTok{strata}\NormalTok{(ID), }\AttributeTok{data =}\NormalTok{ ramlxray) }\SpecialCharTok{\%\textgreater{}\%}
  \FunctionTok{summary}\NormalTok{()}
\end{Highlighting}
\end{Shaded}

\begin{verbatim}
## Call:
## coxph(formula = Surv(rep(1, 223L), disease) ~ Sex + Fray + unclass(CnRay) + 
##     strata(ID), data = ramlxray, method = "exact")
## 
##   n= 223, number of events= 104 
## 
##                   coef exp(coef) se(coef)     z Pr(>|z|)    
## SexM           0.06865   1.07106  0.37448 0.183 0.854548    
## Frayyes        0.66698   1.94834  0.34485 1.934 0.053102 .  
## unclass(CnRay) 0.81144   2.25116  0.23674 3.428 0.000609 ***
## ---
## Signif. codes:  0 '***' 0.001 '**' 0.01 '*' 0.05 '.' 0.1 ' ' 1
## 
##                exp(coef) exp(-coef) lower .95 upper .95
## SexM               1.071     0.9337    0.5141     2.231
## Frayyes            1.948     0.5133    0.9911     3.830
## unclass(CnRay)     2.251     0.4442    1.4154     3.580
## 
## Concordance= 0.654  (se = 0.056 )
## Likelihood ratio test= 19.58  on 3 df,   p=2e-04
## Wald test            = 14.15  on 3 df,   p=0.003
## Score (logrank) test = 17.57  on 3 df,   p=5e-04
\end{verbatim}

\newpage

1 (2 pts). Explain why the variable \texttt{age} is not included in the
model.

2 (3 pts). What is the odds ratio for the effect of the father having
exposure to X-ray against the father having no exposure to X-ray
(variable \texttt{Fray}), controlling for the effect of other
predictors?

3 (2 pts). Explain why the function for fitting a conditional logistic
regression is provided by a package for survival analysis.

4 (2 pts). Explain when conditional logistic regression should be used
over unconditional logistic regression.

\newpage

\subsection{Q4: (Extra Space)}\label{q4-extra-space}

\newpage

\subsection{Q5. (2 pts) True or false}\label{q5.-2-pts-true-or-false}

\begin{longtable}[]{@{}
  >{\raggedright\arraybackslash}p{(\columnwidth - 6\tabcolsep) * \real{0.0645}}
  >{\raggedright\arraybackslash}p{(\columnwidth - 6\tabcolsep) * \real{0.0645}}
  >{\raggedright\arraybackslash}p{(\columnwidth - 6\tabcolsep) * \real{0.0645}}
  >{\raggedright\arraybackslash}p{(\columnwidth - 6\tabcolsep) * \real{0.8065}}@{}}
\toprule\noalign{}
\begin{minipage}[b]{\linewidth}\raggedright
No.
\end{minipage} & \begin{minipage}[b]{\linewidth}\raggedright
\end{minipage} & \begin{minipage}[b]{\linewidth}\raggedright
\end{minipage} & \begin{minipage}[b]{\linewidth}\raggedright
Statement
\end{minipage} \\
\midrule\noalign{}
\endhead
\bottomrule\noalign{}
\endlastfoot
1. & T & F & An outcome variable with categories North, South, East, and
West is an ordinal variable. \\
2. & T & F & In a polytomous (or multinomial) logistic regression in
which the outcome variable has five levels, there will be four
intercepts. \\
3. & T & F & In a polytomous (or multinomial) logistic regression in
which the outcome variable has five levels, each independent variable
will have one estimated coefficient. \\
4. & T & F & In a polytomous (or multinomial) model, the decision of
which outcome category is designated as the reference has no bearing on
the parameter estimates since the choice of reference category is
arbitrary. \\
\end{longtable}

\newpage

\subsection{Q6. (13 pts) Multinomial
model}\label{q6.-13-pts-multinomial-model}

Suppose the following polytomous (or multinomial) model is specified for
assessing the effects of \texttt{age} (continuous), \texttt{gender}
(male = 1, female = 0), \texttt{smoke} (smoker = 1, nonsmoker = 0), and
hypertension status \texttt{hpt} (yes = 1, no = 0) on a disease variable
with four outcomes (D = 0 for none, D = 1 for mild, D = 2 for severe,
and D = 3 for critical). \[
\log \left[ \frac{\mathbb{P}(D=g \mid \mathbf{x})}{\mathbb{P}(D=0 \mid \mathbf{x})} \right] = \beta_{g,0} + \text{age} \cdot \beta_{g,\text{age}} + \text{gender} \cdot \beta_{g,\text{gender}} + \text{smoke} \cdot \beta_{g,\text{smoke}} + \text{hpt} \cdot \beta_{g,\text{hpt}},
\] where \(g = 1, 2, 3\).

1 (2 pts). What is the total number of parameters in this model?

2 (3 pts). Give an expression for the odds (severe vs.~none) for a
40-year-old non-smoking male in terms of the regression coefficients.

3 (3 pts). Given an expression for the odds ratio for male vs.~female,
comparing mild disease to none, while controlling for \texttt{age},
\texttt{smoke}, and \texttt{hpt}.

4 (3 pts). Describe how you would perform a likelihood ratio test to
simultaneously test the significance of the \texttt{smoke} and
\texttt{hpt} coefficients. State the null hypothesis, the test
statistic, and the distribution of the test statistic under the null
hypothesis.

5 (2 pts). If we extend the model to allow for an interaction between
\texttt{age} and \texttt{gender} and between \texttt{smoke} and
\texttt{gender}, how many additional parameters would be added to the
model?

\newpage

\subsection{Q6: (Extra Space)}\label{q6-extra-space}

\newpage

\subsection{Q7. (5 pts) True or false}\label{q7.-5-pts-true-or-false}

\begin{longtable}[]{@{}
  >{\raggedright\arraybackslash}p{(\columnwidth - 6\tabcolsep) * \real{0.0645}}
  >{\raggedright\arraybackslash}p{(\columnwidth - 6\tabcolsep) * \real{0.0645}}
  >{\raggedright\arraybackslash}p{(\columnwidth - 6\tabcolsep) * \real{0.0645}}
  >{\raggedright\arraybackslash}p{(\columnwidth - 6\tabcolsep) * \real{0.8065}}@{}}
\toprule\noalign{}
\begin{minipage}[b]{\linewidth}\raggedright
No.
\end{minipage} & \begin{minipage}[b]{\linewidth}\raggedright
\end{minipage} & \begin{minipage}[b]{\linewidth}\raggedright
\end{minipage} & \begin{minipage}[b]{\linewidth}\raggedright
Statement
\end{minipage} \\
\midrule\noalign{}
\endhead
\bottomrule\noalign{}
\endlastfoot
1. & T & F & The disease categories absent, mild, moderate, and severe
can be ordinal. \\
2. & T & F & In an ordinal logistic regression (proportional odds model)
in which the outcome variable has five levels, there will be four
intercepts. \\
3. & T & F & In an ordinal logistic regression in which the outcome
variable has five levels, each independent variable will have four
estimated coefficients. \\
4. & T & F & If the outcome D has seven levels (coded 1, 2,\ldots, 7),
an assumption of the proportional odds model is that
\(\frac{\mathbb{P}(D \le 3)}{\mathbb{P}(D > 3)}\) is assumed equal to
\(\frac{\mathbb{P}(D \le 5)}{\mathbb{P}(D > 5)}\). \\
5. & T & F & If the outcome D has seven levels (coded 1, 2,\ldots, 7)
and a predictor \(E\) has two levels (coded 0 and 1), then an assumption
of the proportional odds model is that
\(\frac{\mathbb{P}(D \le 3 \mid E = 1) / \mathbb{P}(D > 3 \mid E = 1)}{\mathbb{P}(D \le 3 \mid E = 0) / \mathbb{P}(D > 3 \mid E = 0)}\)
is assumed equal to
\(\frac{\mathbb{P}(D \le 5 \mid E = 1) / \mathbb{P}(D > 5 \mid E = 1)}{\mathbb{P}(D \le 5 \mid E = 0) / \mathbb{P}(D > 5 \mid E = 0)}\). \\
6. & T & F & If the outcome D has four categories coded
\(D = 0, 1, 2, 3\), then the log odds of \(D \le 2\) is greater than the
log odds of \(D \le 1\). \\
7. & T & F & Suppose a four level outcome \(D\) coded \(D = 0, 1, 2, 3\)
is recoded \(D^* = 1, 2, 7, 29\), then the choice of using \(D\) or
\(D^*\) as the outcome in a proportional odds model has no effect on the
parameter estimates as long as the order in the outcome is preserved. \\
8. & T & F & In a GLM, the mean response is modeled as linear with
respect to the regression parameters. \\
9. & T & F & In a GLM, a function of the mean response is modeled as
linear with respect to the regression parameters. That function is
called the link function. \\
10. & T & F & Quasi-likelihood estimates may be obtained even if the
distribution of the response variable is unknown. What should be
specified is a function relating the variance to the mean response. \\
\end{longtable}

\newpage

\subsection{Q8. (15 pts) Ordinal logistic
model}\label{q8.-15-pts-ordinal-logistic-model}

Consider the model in Q6. This time we treat the outcome as an ordinal
variable and use a proportional odds model. \[
\log \left[ \frac{\mathbb{P}(D \le g \mid \mathbf{x})}{\mathbb{P}(D > g \mid \mathbf{x})} \right] = \theta_{g} - \text{age} \cdot \beta_{\text{age}} - \text{gender} \cdot \beta_{\text{gender}} - \text{smoke} \cdot \beta_{\text{smoke}} - \text{hpt} \cdot \beta_{\text{hpt}},
\] where \(g =0, 1, 2, 3\).

1 (2 pts). What is the total number of parameters in this model?

2 (5 pts). Give an expression for the odds of a severe or critical
outcome (\(D \ge 2\)) for a 40-year-old male smoker without
hypertension.

3 (3 pts). Give an expression for the mild, severe, or critical stage of
disease (i.e., \(D \ge 1\)) comparing hypertensive smokers vs
nonhypertensive nonsmokers, controlling for \texttt{age} and
\texttt{gender}.

4 (5 pts). Give an expression for the odds ratio for critical disease
only (\(D = 3\)) comparing hypertensive smokers vs nonhypertensive
nonsmokers, controlling for \texttt{age} and \texttt{gender}.

5 (3 pts). Given an expression for the odds ratio for mild or no disease
(\(D < 2\)) comparing hypertensive smokers vs.~nonhypertensive
nonsmokers, controlling for \texttt{age} and \texttt{gender}.

\newpage

\subsection{Q8: (Extra Space)}\label{q8-extra-space}

\newpage

\subsection{Q9. (8 pts) Link functions}\label{q9.-8-pts-link-functions}

Write down the (1) names, (2) expressions, and (3) the name of
corresponding latent variable distribution of 3 commonly used link
functions for a Bernoulli or binomial parameter \(p\).

Example: Identiy link, \(\eta = g(p) = p\), corresponds to a uniform
distribution for the latent variable.

\newpage

\subsection{Q10. (10 pts) Inverse
Gaussian}\label{q10.-10-pts-inverse-gaussian}

Recall in GLM, the distribution of \(Y\) is from the exponential family
of distributions if it takes the form as \[
  f(y \mid \theta, \phi) = \exp \left[ \frac{y \theta - b(\theta)}{a(\phi)} + c(y, \phi) \right].
\]

The inverse Gaussian distribution \(IG(\mu, \lambda)\) has density \[
f(y) = \left( \frac{\lambda}{2 \pi y^3} \right)^{1/2} e^{- \frac{\lambda (y - \mu)^2}{2 \mu^2 y}}, \quad y, \mu, \lambda > 0.
\]

1 (2 pts). We we can rewrite \(IG(\mu, \lambda)\) as the following form

\[
f(y) = \exp \bigg[ \frac{y (-\mu^{-2}) + 2/\mu}{2/\lambda} - \frac{\lambda}{2y} + \frac{\log \lambda - \log (2\pi y^3)}{2}. \bigg]
\] \quad  Does \(IG(\mu, \lambda)\) belong to the exponential family
distributions? If yes, write down expressions for \(\theta\),
\(b(\theta)\), \(a(\phi)\), and \(c(y, \phi)\). If not, explain why not.

If your answer is ``Yes'' to question 1, answer the following questions:

2 (2 pts). What is the canonical parameters?

3 (4 pts). Derive the mean and variance of inverse Gaussian.

4 (2 pts). What is the canonical link function for inverse Gaussian?

\newpage

\subsection{Q10: (Extra Space)}\label{q10-extra-space}

\newpage

\subsection{\texorpdfstring{Appendix: \(\chi^2\)
Table}{Appendix: \textbackslash chi\^{}2 Table}}\label{appendix-chi2-table}

\begin{verbatim}
## 
## Attaching package: 'kableExtra'
\end{verbatim}

\begin{verbatim}
## The following object is masked from 'package:dplyr':
## 
##     group_rows
\end{verbatim}

\begin{table}[H]
\centering
\caption{\label{tab:unnamed-chunk-8}Chi Square Critical Value Table.}
\centering
\resizebox{\ifdim\width>\linewidth\linewidth\else\width\fi}{!}{
\begin{tabular}[t]{ccccccccc}
\toprule
\multicolumn{1}{c}{ } & \multicolumn{8}{c}{P-value} \\
\cmidrule(l{3pt}r{3pt}){2-9}
Df & 0.99 & 0.9 & 0.1 & 0.05 & 0.02 & 0.01 & 0.005 & 0.001\\
\midrule
\cellcolor{gray!10}{1} & \cellcolor{gray!10}{0.0002} & \cellcolor{gray!10}{0.0158} & \cellcolor{gray!10}{2.7055} & \cellcolor{gray!10}{3.8415} & \cellcolor{gray!10}{5.4119} & \cellcolor{gray!10}{6.6349} & \cellcolor{gray!10}{7.8794} & \cellcolor{gray!10}{10.8276}\\
2 & 0.0201 & 0.2107 & 4.6052 & 5.9915 & 7.8240 & 9.2103 & 10.5966 & 13.8155\\
\cellcolor{gray!10}{3} & \cellcolor{gray!10}{0.1148} & \cellcolor{gray!10}{0.5844} & \cellcolor{gray!10}{6.2514} & \cellcolor{gray!10}{7.8147} & \cellcolor{gray!10}{9.8374} & \cellcolor{gray!10}{11.3449} & \cellcolor{gray!10}{12.8382} & \cellcolor{gray!10}{16.2662}\\
4 & 0.2971 & 1.0636 & 7.7794 & 9.4877 & 11.6678 & 13.2767 & 14.8603 & 18.4668\\
\cellcolor{gray!10}{5} & \cellcolor{gray!10}{0.5543} & \cellcolor{gray!10}{1.6103} & \cellcolor{gray!10}{9.2364} & \cellcolor{gray!10}{11.0705} & \cellcolor{gray!10}{13.3882} & \cellcolor{gray!10}{15.0863} & \cellcolor{gray!10}{16.7496} & \cellcolor{gray!10}{20.5150}\\
6 & 0.8721 & 2.2041 & 10.6446 & 12.5916 & 15.0332 & 16.8119 & 18.5476 & 22.4577\\
\cellcolor{gray!10}{7} & \cellcolor{gray!10}{1.2390} & \cellcolor{gray!10}{2.8331} & \cellcolor{gray!10}{12.0170} & \cellcolor{gray!10}{14.0671} & \cellcolor{gray!10}{16.6224} & \cellcolor{gray!10}{18.4753} & \cellcolor{gray!10}{20.2777} & \cellcolor{gray!10}{24.3219}\\
8 & 1.6465 & 3.4895 & 13.3616 & 15.5073 & 18.1682 & 20.0902 & 21.9550 & 26.1245\\
\cellcolor{gray!10}{9} & \cellcolor{gray!10}{2.0879} & \cellcolor{gray!10}{4.1682} & \cellcolor{gray!10}{14.6837} & \cellcolor{gray!10}{16.9190} & \cellcolor{gray!10}{19.6790} & \cellcolor{gray!10}{21.6660} & \cellcolor{gray!10}{23.5894} & \cellcolor{gray!10}{27.8772}\\
10 & 2.5582 & 4.8652 & 15.9872 & 18.3070 & 21.1608 & 23.2093 & 25.1882 & 29.5883\\
\cellcolor{gray!10}{11} & \cellcolor{gray!10}{3.0535} & \cellcolor{gray!10}{5.5778} & \cellcolor{gray!10}{17.2750} & \cellcolor{gray!10}{19.6751} & \cellcolor{gray!10}{22.6179} & \cellcolor{gray!10}{24.7250} & \cellcolor{gray!10}{26.7568} & \cellcolor{gray!10}{31.2641}\\
12 & 3.5706 & 6.3038 & 18.5493 & 21.0261 & 24.0540 & 26.2170 & 28.2995 & 32.9095\\
\cellcolor{gray!10}{13} & \cellcolor{gray!10}{4.1069} & \cellcolor{gray!10}{7.0415} & \cellcolor{gray!10}{19.8119} & \cellcolor{gray!10}{22.3620} & \cellcolor{gray!10}{25.4715} & \cellcolor{gray!10}{27.6882} & \cellcolor{gray!10}{29.8195} & \cellcolor{gray!10}{34.5282}\\
14 & 4.6604 & 7.7895 & 21.0641 & 23.6848 & 26.8728 & 29.1412 & 31.3193 & 36.1233\\
\cellcolor{gray!10}{15} & \cellcolor{gray!10}{5.2293} & \cellcolor{gray!10}{8.5468} & \cellcolor{gray!10}{22.3071} & \cellcolor{gray!10}{24.9958} & \cellcolor{gray!10}{28.2595} & \cellcolor{gray!10}{30.5779} & \cellcolor{gray!10}{32.8013} & \cellcolor{gray!10}{37.6973}\\
16 & 5.8122 & 9.3122 & 23.5418 & 26.2962 & 29.6332 & 31.9999 & 34.2672 & 39.2524\\
\cellcolor{gray!10}{17} & \cellcolor{gray!10}{6.4078} & \cellcolor{gray!10}{10.0852} & \cellcolor{gray!10}{24.7690} & \cellcolor{gray!10}{27.5871} & \cellcolor{gray!10}{30.9950} & \cellcolor{gray!10}{33.4087} & \cellcolor{gray!10}{35.7185} & \cellcolor{gray!10}{40.7902}\\
18 & 7.0149 & 10.8649 & 25.9894 & 28.8693 & 32.3462 & 34.8053 & 37.1565 & 42.3124\\
\cellcolor{gray!10}{19} & \cellcolor{gray!10}{7.6327} & \cellcolor{gray!10}{11.6509} & \cellcolor{gray!10}{27.2036} & \cellcolor{gray!10}{30.1435} & \cellcolor{gray!10}{33.6874} & \cellcolor{gray!10}{36.1909} & \cellcolor{gray!10}{38.5823} & \cellcolor{gray!10}{43.8202}\\
20 & 8.2604 & 12.4426 & 28.4120 & 31.4104 & 35.0196 & 37.5662 & 39.9968 & 45.3147\\
\bottomrule
\end{tabular}}
\end{table}

\end{document}
